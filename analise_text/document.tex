\documentclass[]{report}
\usepackage{amssymb,amsfonts,amsmath,mathtext,cite,enumerate,float}
\usepackage[pdftex]{graphicx} 
\graphicspath{{images/}}
\usepackage[T2A]{fontenc}
\usepackage[utf8x]{inputenc}
\usepackage[english, russian]{babel}

% Title Page
\title{Первое задание по курсу NLP}
\author{Малюгина Ольга}


\begin{document}
\maketitle

\begin{abstract}
	В данном задании необходимо исслодовать и применить на практике следующие умения и методы:
	\begin{itemize}
		\item предобработка текстов;
		\item использование словарей;
		\item лемматизация;
		\item применение N-грамм, коллокациий;
		\item применение матричных разложений;
		\item использование дистрибуционной семантики;
		\item взвешивание признаков
		\item применение классификаторов;
		\item построение ансамблей различных классификаторов и методов;		
	\end{itemize}
	Все операции производились над четырмя массивами данных.
\end{abstract}

\section{polarity dataset}
В датасете представлены 1000 позитивных и 1000 негативных рецензий.Необходимо постоить классификатор, который по рецензии определяет ее настроение.
\subsection{предобработка текстов}
Вначале удаляем все лишние символы (запятые, скобки, точки и т.д.).

\end{document}          
